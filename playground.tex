
\documentclass{exam}
\usepackage{multicol}
 \usepackage{amssymb} 
  \usepackage{tabularx} 

 \usepackage[most]{tcolorbox}
 \usepackage{xhfill}
 \usepackage{enumitem}

 
 
 
 
 \newcommand{\arulefill}{%
 	\phantom{a}\xrfill[-1ex]{0.4pt}[black]\phantom{a}
 }
 
 \makeatletter
 \renewcommand \dotfill {\leavevmode \cleaders \hb@xt@ 2em{\hss .\hss }\hfill \kern \z@}
 \makeatother
 
 \usepackage{tikz}
 \usetikzlibrary{patterns}
 
 
 %%-----------------------------------------------------------------------
 %% This section sets up a routine for filling the squares in a
 %% grid with null lines.
 %%-----------------------------------------------------------------------
 \def\squaresize{0.25in}
 \pgfdeclarepatternformonly
 {lightcones}% name
 {\pgfpointorigin}% lower left
 {\pgfpoint{\squaresize}{\squaresize}}%  upper right
 {\pgfpoint{\squaresize}{\squaresize}}%  tile size
 {% shape description
 	\pgfsetlinewidth{0.4pt}
 	\pgfpathmoveto{\pgfpoint{0in}{0in}}
 	\pgfpathlineto{\pgfpoint{\squaresize}{\squaresize}}
 	\pgfpathmoveto{\pgfpoint{0in}{\squaresize}}
 	\pgfpathlineto{\pgfpoint{\squaresize}{0in}}
 	\pgfusepath{stroke}
 }
 
 %%-----------------------------------------------------------------------
 %% This section sets up a routine for filling a region with dots
 %%-----------------------------------------------------------------------
 % Re-use the quantity \squaresize defined above
 \def\dotsize{.7pt}
 \pgfdeclarepatternformonly
 {dotgrid}% name
 {\pgfpoint{-0.5*\squaresize}{-0.5*\squaresize}}% lower left
 {\pgfpoint{0.5*\squaresize}{0.5*\squaresize}}%  upper right
 {\pgfpoint{\squaresize}{\squaresize}}%  tile size
 {% shape description
 	\pgfpathcircle{\pgfqpoint{0pt}{0pt}}{\dotsize}
 	\pgfusepath{fill}
 }


	 
\begin{document}
	
	kjsdklsgjfkl fsjgklfdjglk j
	
	dfgfd
	gfdg
	dfsg
	dfg
	df
	d
	g
	dg
	

\tcbset{
	posterbox/.style={%
		enhanced jigsaw, size=fbox,
		colback=#1!10, colframe=#1!10!black,
		colbacktitle=#1!70!black
	},
}

\begin{tcbposter}[
	coverage={spread outwards,fill downwards,spread inwards,phantom=\thispagestyle{empty}},
	poster={columns=3,rows=5,showframe=false,spacing=6mm
},
	fontsize=12pt,
boxes={enhanced,	
	top=8mm,
		colframe=blue!50!black,colback=blue!10!white,colbacktitle=blue!5!yellow!10!white,
	fonttitle=\bfseries,coltitle=black,attach boxed title to top center=
	{yshift=-0.25mm-\tcboxedtitleheight/2,yshifttext=2mm-\tcboxedtitleheight/2},
	boxed title style={boxrule=0.5mm,
		frame code={ \path[tcb fill frame] ([xshift=-4mm]frame.west)
			-- (frame.north west) -- (frame.north east) -- ([xshift=4mm]frame.east)
			-- (frame.south east) -- (frame.south west) -- cycle; },
		interior code={ \path[tcb fill interior] ([xshift=-2mm]interior.west)
			-- (interior.north west) -- (interior.north east)
			-- ([xshift=2mm]interior.east) -- (interior.south east) -- (interior.south west)
			-- cycle;} }}
		]
	
		\posterbox[title=Work Priorities]{row=2}{    
	\begin{enumerate}
		%	  \setlength\itemsep{0.3*}
		\setlength\itemsep{(\tcbtextheight/5)}
		\item \arulefill
		\item \arulefill
		\item \arulefill
	\end{enumerate}
}

	\posterbox[title=Personal Priorities]{row=2,column=2}{    
	\begin{enumerate}
		%	  \setlength\itemsep{0.3*}
		\setlength\itemsep{(\tcbtextheight/5)}
		\item \arulefill
		\item \arulefill
		\item \arulefill
	\end{enumerate}
}
	\posterbox[title=Habits]{row=2,column=3}{    
	\begin{enumerate}
		%	  \setlength\itemsep{0.3*}
		\setlength\itemsep{(\tcbtextheight/5)}
		\item \arulefill
		\item \arulefill
		\item \arulefill
	\end{enumerate}
}

\posterbox[title=I am looking forward to..., top=3mm, left=0mm, right=0mm,
			underlay={            
	\begin{tcbclipinterior}
%		\coordinate (X) at ([xshift=0,yshift=-5mm]frame.north west);
%		\draw[help lines, ystep=\baselineskip, xstep=\linewidth] (X) grid (frame.south east);
      \fill [pattern=dotgrid] (0,0) rectangle (8.5in,11in);

	\end{tcbclipinterior}
}
]{span=3,row=3,}{}

\posterbox[title=What am I grateful for today?, top=3mm, left=0mm, right=0mm]{span=3,row=4,}
{\fillwithdottedlines{\tcbtextheight}}

	
	
	\posterbox[title=I am looking forward to..., top=3mm, left=0mm, right=0mm]{span=3,row=5,}
		{\fillwithdottedlines{\tcbtextheight}}
		
% 		\posterbox[title=Today I am open to the possibility of..., top=3mm, left=0mm, right=0mm]{span=3,row=6,}	{\fillwithdottedlines{\tcbtextheight}}
	
%		\posterbox[title=Todos and priorities2,		colback=white,			colframe=white,		colbacktitle=white, 		coltitle=black 		]{span=3,row=4}		{\fillwithdottedlines{\tcbtextheight}}


	
\end{tcbposter}


\newpage

	
	\begin{tcbposter}[
		coverage={spread,phantom=\thispagestyle{empty}},
		poster={columns=3,rows=8,spacing=3mm,showframe=false},
		fontsize=12pt,
		]
		
		\posterbox[title=Top left box,posterbox=blue]{name=A,column=1,below=top,span=2,rowspan=1}
		{This is some other text}
		
		\posterbox[title=Bottom right box,posterbox=blue]{name=B,column=2,above=bottom,span=2,rowspan=1}
		{This is some other text
	}
		
		\posterbox[title=Bottom left box,posterbox=green]{name=C,column=1,between=A and bottom}
		{This is some other text}
		
		\posterbox[title=Central box,posterbox=red]{name=D,column=2,between=A and B}
		{\fillwithdottedlines{\tcbtextheight}}
		
		\posterbox[title=Top right box,posterbox=green]{name=E,column=3,between=top and B}
		{This is some other text}
		
	\end{tcbposter}

\newpage


	Please justify all your answers to the following questions.
	\begin{enumerate}
		\item Is this = that?\hfill 2 points
		\begin{tcolorbox}
		\end{tcolorbox}
		\item Is this = that?\hfill 2 points
		\begin{tcolorbox}[height fill]
		\end{tcolorbox}
	\end{enumerate}
	
	%Following line creates a new group with different size
	\tcbset{equal height group=B, minimum for equal height group=B:3cm}
	
	\begin{enumerate}
		\item Is this = that?\hfill 2 points
		\begin{tcolorbox}
		\end{tcolorbox}
		\item Is this = that?\hfill 2 points
		\begin{tcolorbox}
		\end{tcolorbox}
		\item Is this = that?\hfill 2 points
		\begin{tcolorbox}
		\end{tcolorbox}
		\item Is this = that?\hfill 2 points
		\begin{tcolorbox}[height fill]
				\fillwithlines{\tcbtextheight}
		\end{tcolorbox}
	\end{enumerate}

	
	\newpage

\begin{minipage}[s]{\linewidth}
	one
	\vfill
	\end{minipage}
\begin{minipage}[s]{\linewidth}
	two
		\vfill
\end{minipage}


\newpage
	
	
\begin{tcbraster}[raster equal height=rows,
	raster every box/.style={colframe=red!50!black,colback=red!10!white},
	height=6in
]
	\begin{tcolorbox}[blankest,height fill,raster columns=1]
		\begin{tcbraster}[raster columns=1,height=5in,raster equal height=rows]
			\begin{tcolorbox}[height fill=maximum]One\end{tcolorbox}
				\begin{tcolorbox}[height fill=maximum]raster+tcolorbox+raster\end{tcolorbox}
					\begin{tcolorbox}[height fill=maximum]raster+tcolorbox+raster\end{tcolorbox}
		\end{tcbraster}
	\end{tcolorbox}
	\begin{tcolorbox}raster+tcolorbox+raster\end{tcolorbox}
\end{tcbraster}


\newpage
	
The magic word is \fillin

\checkboxchar{$\Box$}

\begin{questions}
	\question One of these things is not like the others; one of these
	things is not the same. Which one is different?
	\begin{oneparcheckboxes}
		\choice John
		\choice Paul
		\choice George
		\choice Ringo
		\CorrectChoice Socrates
	\end{oneparcheckboxes}

\question
In no more than one paragraph, explain why the earth is round.
\makeemptybox{1in}


In no more than one paragraph, explain why the earth is round.
\makeemptybox{\stretch{1}}

What changes to the van Allen radiation belt are needed to make
the earth into a regular icosahedron?
\fillwithlines{1in}

\setlength\linefillheight{.5in}

\question What changes to the van Allen radiation belt are needed to make
the earth into a regular icosahedron?
\fillwithlines{\stretch{1}}

These are dots
\fillwithdottedlines{1in}

\begin{multicols}{2}
	Column 1
\fillwithlines{\stretch{1}}
	\columnbreak
	Column 2
	\fillwithlines{\stretch{1}}
\end{multicols}


\end{questions}

\newpage



\end{document}
